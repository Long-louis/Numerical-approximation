\section{习题一 \ 舍入误差 $\epsilon$ }
\begin{enumerate}
    \item 最终计算得到的$f(x)$结果为0。出现该结果是由于在计算过程中,计算机位数长度的限制。$10^{-16}+1$在计算机内四舍五入为1,再减$1$变为0。
    \item \begin{python}
              for n in range(101, 0, -1):
              if (0.5 + 2 ** (-n)) > 0.5:
              print(n)
              break
          \end{python}
          最终输出结果为$n=53$。
\end{enumerate}

\section{习题二 \ 逼近指数函数}
\begin{figure}[htbp]
    \centering
    \subfigure[$x=10$]
    {
        \begin{minipage}[b]{.4\linewidth}
            \centering
            \includegraphics[scale=0.29]{pic/x=10.png}
        \end{minipage}
    }
    \subfigure[$x=-10$]
    {
        \begin{minipage}[b]{.4\linewidth}
            \centering
            \includegraphics[scale=0.29]{pic/x=-10.png}
        \end{minipage}
    }
    \caption{相对误差}
\end{figure}
\begin{enumerate}
    \item $x=10$时相对误差逐渐下降,到$N=20$之后误差趋于稳定。
    \item $x=-10$时相对误差出现了波峰,直到$N=20$之后误差才趋于稳定,说明此逼近算法不够稳定。
    \item 可以采用插值算法对函数进行逼近。
\end{enumerate}

\section{习题三 \ 计算积分}
利用分部积分,可以得到
\begin{align}
    I_n & = \left[\left.xe^x\right|^{1}_{0} - \int^{1}_{0}nx^{n-1}e^xdx \right] \\
        & = \left[e^1 - n\int^{1}_{0}x^{n-1}e^xdx\right] \notag                 \\
        & = e-nI_{n-1} \notag
\end{align}
此即为递推公式。以$I_0=e-1$为初始值,由递推公式可得$I_{20}=-129.26370813285942$。

$e^x$的级数展开式为$\sum_{n=0}^{\infty}\frac{x^n}{n!}$,将此展开式带入$I_n$中,将求和号与积分号换序,即可得到
\begin{equation}
    I_N=\sum_{n=0}^{\infty} \frac{1}{n !(N+1+n)}, \quad \forall x \in \mathbb{R}
\end{equation}
利用python进行累加求和可得$I_{20}=0.12380383076256994$。

由$I_{20}$的准确值即可验证方法二更加适合进行计算。
\section{习题四 \ 序列是否收敛}

计算可得$U_0=\frac{11}{2},U_1=\frac{61}{11}$,结合递推关系式可得$\lim _{n \rightarrow \infty} u_n=100$。

首先取$(\alpha, \beta, \gamma)=(0,1,1)$,得到

\begin{equation}
    u_n=\frac{\beta 6^{n+1}+\gamma 5^{n+1}}{\beta 6^n+\gamma 5^n}
\end{equation}
于是$\lim _{n \rightarrow \infty} U_n=6$。

针对迭代公式(1),100是一个稳定的收敛点,而6和5是不稳定的收敛点,对误差进行分析:
\begin{equation}
    \begin{aligned}
          & 111-\frac{1130-\frac{3000}{u+\delta}}{u+\delta} \\
          &=  111-1130(u+\delta)^{-1}+3000(u+\delta)^{-2} .
    \end{aligned}
\end{equation}
同时对$(u+\delta)^{-2}$进行Taylor展开有:
\begin{equation}
    \begin{aligned}
    &(u+\delta)^{-1}=\frac{1}{u}-\frac{1}{u^2} \delta-\frac{1}{u^3} \delta^2, \\
    &(u+\delta)^{-2}=\frac{1}{u^2}-\frac{2}{u^3} \delta-\frac{3}{2 u^4} \delta^2 .
    \end{aligned}
    \end{equation}
然后就有
\begin{equation}
    \begin{aligned}
    &111-1130(u+\delta)^{-1}+3000(u+\delta)^{-2} \\
    & \approx -1130 u^{-1}+3000 u^{-2}+\left(\frac{1130}{u^2}-\frac{6000}{u^3}\right) \delta .
    \end{aligned}
    \end{equation}
所以最终产生了误差。